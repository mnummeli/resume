\documentclass[a4paper,12pt]{article}
\usepackage[finnish]{babel}
\usepackage[utf8]{inputenc}
\usepackage[T1]{fontenc}
\usepackage{lmodern}
\usepackage{eurosym}
\usepackage[pdftex,colorlinks]{hyperref}
\title{Curriculum vitae}
\author{Mikko Nummelin}
\date{25.07.2025}
\begin{document}
\maketitle
\section*{Perustiedot}
Nimi: Mikko Ilmari Nummelin \\
Syntymäaika: 25.02.1975 \\
Kansalaisuus: Suomen
\section*{Esittely}

Olen diplomi-insinööri, valmistunut Teknillisestä Korkeakoulusta, nykyisestä Aalto-yliopistosta Teknillisen fysiikan koulutusohjelmasta, pääaineena teknillinen matematiikka, sivuaineena ohjelmistojärjestelmät. Minulla on useamman
vuosikymmenen kokemus ohjelmistotekniikasta, erityisesti verkkoteknologioista ja useista tietokantajärjestelmistä, sekä relaatio- että ei-relaatiokannoista. Olen eniten käyttänyt projekteissa ohjelmointikielenä Javaa, mutta minulla on myös kokemusta Pythonista ja Javascriptistä sekä niitä käyttävistä ohjelmistokirjastoista, kuten Angular:ista, ReactJS:stä ja JQuery:stä. Lisäksi olen tehnyt
asiakasprojekteja, joissa on käytetty Google Cloud ja Microsoft Azure-pilvipalveluja.

Minulla on henkilökohtainen GitHub-sivu harrastusasioista osoitteessa:\\
\href{http://github.com/mnummeli}{http://github.com/mnummeli}

Olen laatinut vapaa-ajalla joitakin matematiikan ja tietotekniikan kursseja Työväen Sivistysliiton Espoon ja Kauniaisten opintojärjestön sivuille. Toimin myös kyseisen järjestön verkkosivujen ylläpitäjänä. Linkkejä näihin löytyy osoitteessa:\\
\href{http://www.tslespoo.fi}{http://www.tslespoo.fi}

Harrastan lisäksi shakkia ja kuntosalia.
\section*{Koulutus}
\subsection*{Teknillinen korkeakoulu (1994-2007)}
Diplomi-insinööri, teknillinen fysiikka
\subsection*{Etelä-Tapiolan lukio (1991-1994)}
Ylioppilas
\section*{Kielitaidot}
\begin{itemize}
\item{suomi (äidinkieli)}
\item{englanti (hyvä)}
\item{viro (hyvä)}
\item{ruotsi (kohtalainen)}
\item{saksa (jonkin verran)}
\end{itemize}
\section*{Sertifikaatit}
Microsoft Azure Fundamentals (lokakuu 2024)
\section*{Teknologiat}
\subsection*{Ohjelmointikielet, kirjastot ja verkkoteknologiat}
Java, Spring Boot, Javascript, Typescript, Python, HTML, CSS, Angular, React.js
\subsection*{Pilvipalvelut}
Google Cloud Platform, Microsoft Azure
\subsection*{Tietokannat}
MySQL, PostgreSQL, Oracle, Redis, LDAP
\subsection*{Konttiteknologiat}
Docker, Kubernetes, Openshift
\subsection*{Testaus}
Robot Framework, JUnit, Mockito, Selenium
\subsection*{Versionhallinta ja CI/CD}
Git, Mercurial, SVN, Jenkins
\section*{Työhistoria ja -kokemus}
\subsection*{HiQ Finland Oy (02/2023 - nykyhetki)}
\begin{itemize}
\item{Veikkaus Oy: Java, Google Cloud Platform, PostgreSQL}
\item{LähiTapiola: Java, Spring Boot, Spring Framework, Azure}
\item{Vierumäki/St1: Typescript, GCP, Redis}
\end{itemize}
\subsection*{Cinia Oy (01/2015 - 02/2023)}
\begin{itemize}
\item{LMF Ericsson: Java, Vaadin, LDAP, Javascript, OSGi}
\item{Adven Oy: Java}
\item{Suomen Asiakastieto Oy: Java}
\item{EKE: Javascript}
\end{itemize}
\subsection*{Ixonos Oyj (01/2008 - 01/2015)}
\begin{itemize}
\item{Tulli: Java, Spring Framework, BEA Weblogic}
\item{Flexim: Java, MySQL, Struts}
\item{HRX: Java, Apache Tomcat, PostgreSQL}
\end{itemize}
\subsection*{Teknillinen korkeakoulu (2000-2007)}
Osa-aikaisia tehtäviä matematiikan assistenttina 2000-2007, diplomityö, tutkimusapulainen ja tutkija 05/2007-12/2007. Teknologiat: Mathematica, Matlab, LaTeX.
\section*{Varusmiespalvelus}
01/1995-09/1995: Uudenmaan jääkäripataljoona, kranaatinheitinkomppania ja esikuntakomppania, sotilasarvo: jääkäri
\subsection*{Linkkejä}
\href{http://www.mikkonummelin.fi}{http://www.mikkonummelin.fi} \\
\href{http://github.com/mnummeli}{Github-sivustoni} \\
\href{http://www.tslespoo.fi}{http://www.tslespoo.fi}
\end{document}
